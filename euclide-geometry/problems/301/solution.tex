% Solution: 3단계 풀이 통합

\section*{Step 1}
선분 $\overline{AH}$의 연장선이 변 $\overline{BC}$와 만나는 점을 $R$이라 하자.

다음 사실들에 주목한다.
$\angle RHD = 90°$ ,\; $\angle DCR = 90°$ , \; $\angle RQD = 90°$

\section*{Step 2}
네 점 $C$, $D$, $H$, $R$ 이 한 원 위에 있다. (직각 $\angle DCG = 90°$이므로 $\overline{DG}$이 지름)
결국 $C$, $D$, $F$, $H$, $G$ 는 같은 원 위의 점들임을 안다.

원주각을 살피면
$\angle QCP = \angle QDP$ (같은 호에 대한 원주각)
$\angle QDP = \angle HAP$ (엇각 관계)

\section*{Step 3}
따라서 $\angle QCP = \angle HAP$이고, 이를 통해 삼각형의 닮음 관계를 활용하여
$\triangle HCQ$의 넓이를 구할 수 있다.
